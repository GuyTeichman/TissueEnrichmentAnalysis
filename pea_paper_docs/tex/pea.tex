\documentclass[10pt,letterpaper,twocolumn]{article}
% \usepackage[twocolumn]{geometry}
\usepackage{lmodern}% http://ctan.org/pkg/lm

% use Unicode characters - try changing the option
% if you run into troubles with special characters (e.g. umlauts)
\usepackage[utf8x]{inputenc}
% clean citations
\usepackage{cite}
% hyperref makes references clicky. use \url{www.example.com}
% or \href{www.example.com}{description} to add a clicky url
\usepackage{nameref}
% line numbers
\usepackage[right]{lineno}
% improves typesetting in LaTeX
% \usepackage{microtype}
% degree symbol
\usepackage{gensymb}
\usepackage{amsmath}
\usepackage{booktabs}
% use adjustwidth environment to exceed text width (see examples in text)
\usepackage{changepage}

% adjust caption style
\usepackage[aboveskip=1pt,labelfont=bf,
            labelsep=period,singlelinecheck=off]{caption}

% remove brackets from references
\makeatletter
\renewcommand{\@biblabel}[1]{\quad#1.}
\makeatother

% headrule, footrule and page numbers
\usepackage{lastpage,fancyhdr,graphicx}
\usepackage{epstopdf}
\pagestyle{myheadings}
\pagestyle{fancy}
\fancyhf{}
\rfoot{\thepage/\pageref{LastPage}}
\renewcommand{\footrule}{\hrule height 2pt \vspace{2mm}}
% \fancyheadoffset[L]{2.25in}
% \fancyfootoffset[L]{2.25in}

% use \textcolor{color}{text} for colored text (e.g. highlight to-do areas)
\usepackage{color}

% define custom colors (this one is for figure captions)
\definecolor{Gray}{gray}{.25}

% this is required to include graphics
\usepackage{graphicx}

% use if you want to put caption to the side of the figure - see example in text
\usepackage{sidecap}
% \usepackage[urlcolor  = blue]{hyperref}

% hyperurls packages:
\usepackage{xcolor}
\usepackage[colorlinks = true,
            linkcolor = blue,
            urlcolor  = blue,
            citecolor = blue,
            anchorcolor = blue]{hyperref}


% use for have text wrap around figures
\usepackage{wrapfig}
\usepackage[pscoord]{eso-pic}
\usepackage[fulladjust]{marginnote}
\reversemarginpar{}

% new commands
\newcommand{\cel}{\emph{C.~elegans}}
\newcommand{\fog}{\emph{\mbox{fog-2}}}
\newcommand{\ecol}{\emph{E.~coli}}

\newcommand{\hobesity}{957}
\newcommand{\wobesity}{614}

\newcommand{\hlupus}{283}
\newcommand{\wlupus}{135}

\newcommand{\harthritis}{309}
\newcommand{\warthritis}{124}

\newcommand{\qval}[1]{
                      \ensuremath{
                                  q<10^{-#1}
                                  }
                      }

% more space between rows
\newcommand{\ra}[1]{\renewcommand{\arraystretch}{#1}}

\title{
  \Large
  \textbf{
  % Phenotype Enrichment Discovers Phenologs for Disease Modeling in \cel{}
  Phenotype and gene ontology enrichment as guides for disease modeling
  in \cel{}
          }
}

\author{David Angeles-Albores\textsuperscript{1,$\dagger{}$}
\and{}
Paul W. Sternberg\textsuperscript{1,*}
}

\begin{document}

% \vspace*{0.35in}

% title goes here:
\twocolumn[
% title
\maketitle


]
% now start line numbers
\nolinenumbers{}

\section*{Introduction}
The last decade has seen an explosion of techniques capable of genome-wide
measurements. Some examples of genome-wide tools include RNA-seq~\cite{} to
measure gene expression, CHIP-seq~\cite{} to measure binding, or genome-wide
methylation profiling to understand epigenetic changes~\cite{}. These tools are capable
of generating large quantities of data. Understanding these data, and generating
hypotheses from them remains challenging. A common approach used to understand
these datasets is to reduce the dimensionality of the data via enrichment
analyses of ontologies~\cite{}, which helps researchers understand what terms are
overrepresented beyond random levels. By analyzing overrepresented terms in
aggregate, researchers can better understand what biological processes were most
affected in a given experiment, and form hypotheses about what is
happening~\cite{}.
This approach is limited by what ontologies can be tested for enrichment.
The best-known ontology for biological research is the Gene Ontology (GO), which
provides a controlled language to describe molecular and cellular functions of
genes~\cite{}. In \cel{}, curated tissue and phenotype ontologies exist which provide
controlled languages with which to describe \cel{} anatomy and phenotypes
respectively~\cite{}. However, enrichment tools only exist for gene and tissue ontologies
in the community today (see for example~\cite{}). Another limitation is that
tissue enrichment testing is not offered on the same websites as GO enrichment
testing, which requires users to test their data on different websites that may
or may not use different methodologies to detect enrichment.

Another way to use enrichment tools is for evolutionary comparison purposes.
In molecular biology it is often useful to know when a gene is homologous
between two species---that is to say, common by descent---because knowledge of
homology often brings with it knowledge of function~\cite{}. Indeed, many
important gene regulatory networks (GRNs) are conserved between organisms as
highly diverged as nematodes and humans (for example, see~\cite{}). While genes
and GRNs may be conserved between species, their outputs often differ however.
For example, although homeobox genes are known to be important for limb
formation in many animals~\cite{}, these genes do not form limbs in nematodes.
The concept of a phenolog has been put forward to explain relationships between
phenotypes that have the same underlying genetic regulatory network~\cite{}.
Formally, two phenotypes are phenologs of each other if the homologs of the
genes that cause a phenotype in an organism cause a second phenotype in another.

To study a clinically relevant disease in a non-human, an appropriate
model has to be established. A straightforward method towards establishing a
disease model in
\cel{} is to link a disease to a causal gene, then to identify the homologous
gene in \cel{} and then to study the function of the genetic homolog to
extrapolate back to humans. However, this method relies on the existence of
known disease genes and requires that the homolog have a phenotype that can be
reliably identified and studied. A fundamentally different way to establish a
disease model in \cel{} would be to identify the phenologs of the disease to be
studied in \cel{} by identifying disease-associated human genes in an unbiased
manner through genome-wide association studies (GWAS) and identified candidate
homolog genes in \cel{}. The homologs can be used to identify \cel{} disease
phenologs, which can in turn be used as the basis for screens to identify
genes that are associated with that phenolog. Approaches similar to this have
been successfully used in the past to make non-obvious links between phenotypes
in different species~\cite{}.

The concept of a phenolog can also be useful when applied within a species.
In \cel{}, not all phenotypes are equally easy to study. Although genome-wide
measurements can help elucidate the genetic network underlying a phenotype,
devising screens to test which genes are functionally important can be difficult.
A common strategy to study phenotypes that are difficult to screen is to
select an easier-to-screen phenolog, and to test positive hits for the true
phenotype of interest
afterwards~\cite{}. Currently, selection of screening phenotypes is
performed based on researcher experience.
By formalizing phenotype enrichment analysis as a tool with which to analyze
gene sets, researchers should be able to formally establish phenologs, which has
consequences for screen design.

An additional problem with genome-wide
queries of \cel{} states (be they developmental, neuronal, or other) is that
they do not
always have a straightforward interpretation in terms of phenotypes. In these
situations, researchers must rely on intuition to select a phenotype to screen
for. As a result, many hits may go unexplored that would prove fruitful. The
question of how to design a screen that is maximally informative is an important
question that has so far not been addressed within this community.

To facilitate understanding of large datasets, and to make discovery of
phenologs easier, we have completed an enrichment tool suite in WormBase
that allows users to rapidly perform phenotype, tissue and gene ontology
enrichment analyses (PEA, TEA and GEA respectively) on curated \cel{} ontologies
using the same methodology for each one. They are located at
% TODO: include links.
We applied our tools towards the unbiased discovery of phenologues of
multigenic, complex diseases including
% Diseases to study:
systemic lupus erythematosus, obesity and obesity-related traits,  and
rheumatoid arthritis
by using genes associated with these
diseases via genome-wide association studies. We also illustrate the utility of
the complete enrichment suite for finding new relationships in complex data by
analyzing a ciliary neuron transcriptome~\cite{}.

\section*{Methods}
\subsection*{Human disease phenolog identification}
We used the GWAS EBI-NHGRI catalog~\cite{} to extract information on all genome-wide
association studies deposited there. We only selected traits that had $>300$
associated genes. We identified 24 traits that met our criteria. Next, we used
DIOPT~\cite{} to identify candidate homologs for the genes associated with these
traits. Briefly, DIOPT combines a large number of methods for identifying homologs
and returns homolog candidates associated with a compound score. Depending on the score,
homologs can be considered `high', `moderate' or `low' rank, reflecting confidence
in the homology. Many-to-one and one-to-many homology relationships are allowed
in DIOPT, reflecting a mixture of uncertainty and family expansion/reduction.
For our study, we only accepted homolog candidates with `high' or `moderate' scores
and we did not insist on a one-to-one relationship between genes.

After we identified worm homologs for each trait, we reassessed how many traits
still had $>100$ gene candidates, and dropped all traits that had less than this
for our analysis. We identified 18 traits that met this criteria. The gene
lists for each of these 18 traits were then analyzed for gene, tissue and
phenotype enrichment. Tissue enrichment was performed using the WormBase Tissue
Enrichment Analysis tool (TEA)~\cite{}.


\section*{Results}
\subsection*{Developing the WormBase enrichment suite}
% TODO add numbers instead of XXX
We developed the dictionaries for PEA and GEA using the same procedure as was
used for TEA~\cite{}. We generated a dictionary that included terms with at
least 50 annotating genes or more and had a similarity threshold of 0.95 for PEA
(the total number of terms in the dictionary was 251, annotated by 9,169 genes
 for the version XXXXX);\@ and we generated a
dictionary that included terms with at least 50 annotating genes or more and
had a similarity threshold of 0.95 for GEA (the total number of terms in this
dictionary is 271, annotated by 14,636 genes for the version XXXX).
\@ Next, we benchmarked the dictionaries on the same gene
sets as TEA and obtained enrichment of all the expected categories. For example,
on a gene set enriched for embryonic muscle genes~\cite{}, the top two enriched
phenotype terms by q-value were `muscle system morphology variant' and `body
wall muscle thick filament variant'; the top two enriched GO terms were
`myofibril' and `striated muscle dense body'. For all the benchmarking
results, see supplementary information. Having generated and validated our
dictionaries, we proceeded to identify phenologs for several common human
diseases.

\subsection*{Applying the WormBase enrichment suite}
To discover phenologs, we first needed to identify genes that
contribute to a disease in an unbiased manner. One way to discover gene
associations in an unbiased manner is to perform GWSA in human populations.
Therefore, we used the GWAS NHGRI-EBI
Catalog~\cite{} to identify genes associated with human diseases. We found the
best nematode candidate homologs for these genes using DIOPT~\cite{} and applied
our enrichment suite to each of these gene regulatory networks.

\subsubsection*{Obesity-related traits}
Obesity-related traits is a category within the GWAS NHGRI-EBI catalog that
pools studies that have measured obesity and other traits associated with
obesity, such as heart rate, physical activity, hormone levels, body composition
and cholesterol levels. Since this category includes many parameters,
we expected there would be many phenologs. GWAS studies have identified
\hobesity{} genes associated with these traits. Using DIOPT, we found
\wobesity{} homologs for these genes. In total, $341/\wobesity{}$ genes had
at least one phenotype annotation; $548/\wobesity{}$ had at least one gene ontology term
annotation; and $427/\wobesity{}$ had at least one tissue term annotation.

Top results for obesity-related traits included `acetylcholinesterase inhibitor
response variant' (38 genes, \qval{6}),
`neurite morphology variant' (21 genes, \qval{2}),
and `thin' (31 genes, \qval{2}).
Terms involving locomotion were significantly enriched, as were terms involving
body shape and food consumption (\qval{1}). Concomitant with these phenologs was
a tissue enrichment in neuron-related terms. GO enrichment suggested that these
genes are participating in `iron ion binding' (40 genes, \qval{20}) and
`tetrapyrrole binding' (37 genes, \qval{13}).

Tissue and phenotype enrichment therefore suggest that obesity-related traits
may be studied in \cel{} through neuron physiology and function, specifically
with respect to acetylcholinesterase inhibitors. Moreover, GO enrichment
implicates iron and tetrapyrrole binding as metabolic components of
the obesity-related phenologs in \cel{}.

\subsection*{Systemic lupus erythematosus}
Systemic lupus is an autoimmune disease that is believed to be polygenic in
nature~\cite{}. It mainly affects women and is characterized by painful
and swollen joints, hair loss, and fatigue~\cite{}. Since worms do not have a
cellular immune system, we were interested in what phenologs corresponded to
this disorder in \cel{}. To establish phenolog candidates, we obtained
\hlupus{} genes associated with the disease via GWAS studies, and found
\wlupus{} homolog candidates in \cel{}.

Lupus-associated homologs were reasonably well annotated. Slightly more than half
of the genes had at least one phenotype annotation ($76/\wlupus{}$) and almost
all genes were annotated to at least one tissue or gene ontology term
($104/\wlupus{}$ and $115/\wlupus$ genes respectively).
We found that Lupus-associated homologs were enriched in `aneuploidy' (7 genes,
\qval{1}) and `meiotic chromosome segregation' (8 genes, \qval{1}). `Cell fate
transformation' (6 genes, \qval{1}), and `excess intestinal cells'
(5 genes, \qval{1}) were also overrepresented, as was `male tail morphology'
(6 genes, \qval{1}). Finally, the phenotype `nonsense mRNA accumulation' was also
enriched (5 genes, \qval{1}).
Meanwhile, TEA suggested that the
`excretory duct cell' (5 genes, \qval{2}) and the `posterior gonad arm' are
overrepresented in this dataset. We also found that the Pn.p cells P3.p through
P8.p were enriched in this dataset (5 genes, \qval{1}). GO enrichment pointed
at `modification-dependent macromolecule catabolic process' (23 genes,
\qval{15}) as a molecular function that characterizes this dataset. However,
this GO term was enriched only due to a single gene family, the \emph{skr} gene
family. Almost the entire \emph{skr} family was considered a candidate homolog
to the SKP1 human gene, making the GO enrichment suspect.

Enrichment of the terms for `aneuploidy', `meiotic chromosome segregation',
and `excess intestinal cells' were largely driven by the same gene group, which
includes \emph{cki-1}, and several \emph{skr} genes. On the other hand, `cell
fate transformation' and `male tail morphology' reflected the involvement of
developmental genes \emph{let-23}, and \emph{lin-12} among others. The term
`nonsense mRNA accumulation' was the result of \emph{pept-3}, \emph{smg-7},
\emph{tsr-1}, \emph{dhcr-7} and \emph{F08B4.7}. Therefore, we conclude that
systemic lupus erythematosus is potentially represented by a combination of three
phenotypes in \cel{}: A cell proliferation phenotype (either increased or
decreased), probably marked by increased
aneuploidy; a developmental phenotype involving cell fate transformation and
leading to dysmorphias; and a molecular phenotype involving impairment of the
nonsense-mediated decay pathway.
% TODO: Is the text below true???
The results from the tissue enrichment analysis
highlighted three tissues that are particularly sensitive to \emph{lin} mutations
(the gonad, the excretory duct cell and the vulval precursor cells),
and the gonad arms undergo large quantities of nuclear proliferation.

\subsection*{Rheumatoid arthritis}
Rheumatoid arthritis is an auto-immune disease that is characterized by
swollen and painful joints that progressively deteriorate~\cite{}. Unlike lupus,
rheumatoid arthritis is not life-threatening~\cite{}, and comorbidity between rheumatoid
arthritis and lupus is low~\cite{}, suggesting that they may have at least
partially distinct genetic causes. We found \harthritis{} genes associated with
rheumatoid arthritis, for which we found \warthritis{} worm homolog candidates.

% TODO: Is no tissue enrichment observed bc poor annotations, or because
% all tissues randomly represented?
The only phenotype that was enriched for these homologs was `short' (10 genes,
\qval{4}), even though 64 homologs were associated with at least
one phenotype term. No tissue was enriched in this dataset. Because
82 genes are annotated to have expression in at least one tissue, the lack of
enrichment does not reflect ignorance about the sites of expression of these
genes. GEA showed that enriched
molecular functions for these genes include `collagen trimer' (22 genes,
\qval{15}). However, this term was enriched as the result of degeneracy in the
homolog candidates for the SFTPD gene. Other terms included
 `glycosylation' (10 genes, \qval{4}) and `Golgi apparatus' (11
genes, \qval{3}), but this enrichment was also the result of degenerate homolog
candidates for the human gene B3GNT7 which encodes a
beta-1,3-N-acetylgalactosaminyltransferase.

The `short' phenotype was the result of the \emph{cat-4}, \emph{dpy-7},
\emph{rnt-1}, \emph{sem-4}, \emph{unc-116}, \emph{ocrl-1} and some genes in the
\emph{fat} family. Although these genes are bound by a common phenotype, any
genetic relationships between these genes are not immediately clear. Some genes,
like \emph{sem-4} and \emph{rnt-1} are likely transcription factors with roles
in development (including hypodermal development).
Others are molecular motors (\emph{unc-116}) that are broadly expressed throughout
the body of \cel{}. Yet others have known roles in neuron and muscle function,
such as \emph{ocrl-1} and \emph{cat-4}. The `short' phenotype is a subset of
the `body length variant' phenotype. Body length in \cel{} can be controlled via
cell size, shape and number\cite{}; alternatively, cuticle development can alter body
shape~\cite{}; finally, muscles can alter the effective body length artificially~\cite{}.

% TODO: SO.... ?


\subsection*{Ontology Enrichment as an aid for screen design}
An additional use for a tool like PEA would be as a tool to help guide and
design screens to identify genes from an RNA-seq or other genome-wide experiment
for further study. This would be particularly useful in cases when researchers
may not know what phenotype to expect, in which case PEA can guide selection of
a phenotype. Another use case is a scenario where the phenotype under study is
not easy to screen for. By finding phenologs to the phenotype of interest, the
researcher can design an easier screen for genes that affect the phenolog in
question, then re-test genes for the original phenotype of interest.

\subsubsection*{Enrichment in the ciliary neuronal transcriptome}
As an example of how ontology enrichment can improve our understanding of
transcriptomes,
we selected a ciliary neuron dataset~\cite{} and ran the complete WormBase
Enrichment Suite on it. Ciliary neurons are present in the \cel{} male tail, but
they are also present in the male cephalic sensillum and hermaphrodites also have
ciliated neurons. PEA reveals that the ciliated
neuron transcriptome is enriched for genes that are typically associated with
`meiotic chromosome segregation' (46 genes, \qval{5}),
`aneuploidy' (42 genes, \qval{5}) and `spindle defective early embryos' (45
genes, \qval{2}).

In addition, TEA points at the \cel{} gonad primordium, the somatic gonad
and early embryonic cells as the sites where genes associated with ciliary neurons
are enriched. The `male distal tip cell' is a tissue
that is overrepresented in this dataset, but `distal tip cell' is not enriched,
which suggests that structures that are present only in the male tissue are
overrepresented in this dataset.

Although one interpretation of the results would be that microtubule genes
are driving the enrichment of these terms, another possibility is that
there are cell-cycle genes that are driving the enrichment of these phenotypes
and tissues. Indeed, GO enrichment shows terms such as `DNA replication' (29
genes, \qval{5}), and `purine NTP-dependent helicase activity'
(15 genes, \qval{1}). Visual inspection of list in question reveals that
cell-cycle and DNA replication/repair genes are abundant in this transcriptome
and include genes such as \emph{atm-1}, \emph{dna-2}, or \emph{hpr-17}. This
analysis reveals that the ciliary neuron transcriptome is enriched in
genes associated with microtubules, but the cell-cycle machinery is also
carefully regulated.

\subsection*{Deconstructing phenotype-tissue relationships}
\subsubsection*{Tissue enrichment on the Egl gene set reveals cellular
components of the phenotype}
How does a phenotype emerge? We realized that with the tools that we have
developed, it is possible to understand what tissues contribute to a phenotype
in a probabilistic framework. In other words, we can extract all genes associated
with a particular phenotype, then search for tissue terms that are enriched to
understand how a phenotype arises from interactions between anatomical regions.
As a test of this, we selected the egg-laying defective (Egl) phenotype.
In \cel{}, egg-laying is a complex behavior that involves
a large number of tissues~\cite{}. The somatic gonad acts as a repository for the eggs,
the uterine seam cells help protect the uterus, and a variety of muscles help
contract the uterus and open the vulva to lay an egg~\cite{}. The vulva must be well-formed
to allow passage of an egg, and the hermaphrodite-specific neuron (HSN) is involved
in the egg-laying control~\cite{}. The complexity of the interactions that
happen to allow egg-laying make understanding the Egl phenotype in terms of
tissues a challenging task.

We extracted all of the \cel{} genes that have been associated with an Egl
phenotype and we used TEA to understand what tissues are enriched. The HSN
was enriched more than five-fold above background (\qval{7}) as were vulD, vulC,
vulE and vulF (\qval{6}). The vulA, vulB2 and vulB1 were
enriched at slightly lower levels (\qval{5}), whereas the uterine muscles and
uterine seam cells were enriched more than twice above background levels
(\qval{2}).
Therefore, the Egl phenotype would seem to emerge primarily from defects in the
HSN, secondarily from defects in the vulva, and only sometimes from defects in the
uterine seam cells or muscles. It is notable that all vulval cells were not equally
enriched. Although all the `vul' cells are annotated to a similar degree (between
50--70 genes for each cell type), the vulD and vulC cells had the largest
enrichment effect size and the lowest q-values, suggesting that these cells
are more likely to be associated with an Egl phenotype than the others. This may
reflect the fact that vulD and vulE are the site of attachment for four vulval
muscles, vm1. Perhaps this attachment is particularly fragile, and perturbations
to these cells prevent adequate function of these muscles. In support of these
observations, P7.pa had the largest fold-enrichment of any tissue. In \cel{},
P7.pa gives rise to vulD and vulC. However, vulF is
also attached to a set of four additional vulval muscles, vm2. Why
is vulF less associated with an Egl phenotype?

\subsubsection*{Quantifying the anatomy-phenotype
                mapping via Bayesian probabilities}
Another way to understand the phenotype-anatomy mapping is by considering
how informative a given anatomy term is on a particular phenotype, or vice-versa.
To this end, we calculated two conditional probabilities that helped us answer
this question. The first conditional probability,

\begin{equation}
  P( \text{a gene has Egl annotation} | \text{it is expressed in } X)
\end{equation}

answers the question: For a gene with an expression pattern that includes the
tissue term X (i.e., the gene is expressed at least in X), what is the probability
that this gene has an Egl phenotype (i.e., the phenotype annotations for this
gene include Egl)? For simplicity, we can re-write this equation more succintly
by removing a few words. The calculation of this probability is straightforward
and follows from the definition of conditional probability:

\begin{equation}
  P(\text{Egl}|X) = \frac{N_\text{genes annotated Egl and X}}
                         {N_\text{genes annotated with X}}.
\label{egl_x}
\end{equation}

Equation~\ref{egl_x} measures how likely a gene is to be annotated with an Egl
phenotype given that its expression pattern includes the term $X$. A related
quantity (which is neither the inverse nor the complement) is the conditional
probability that a gene which is annotated with at least the Egl phenotype is
expressed in tissue $X$. That is to say,

\begin{equation}
  P(X|\text{Egl}) = \frac{N_\text{genes annotated Egl and X}}
                         {N_\text{genes annotated with Egl}}.
  \label{x_egl}
\end{equation}

Equation~\ref{x_egl} tells us how probable it is that any given gene that is
annotated with an Egl phenotype includes $X$ as a tissue term. Taken together,
equations~\ref{egl_x} and~\ref{x_egl} help us understand how predictive
anatomic expression is of phenotypes, and how predictive phenotypes are of
anatomic expression.

\begin{table*}
  \renewcommand{\familydefault}{\sfdefault}\normalfont{}
  \centering{}
  \ra{1.3}
  \caption{Conditional probabilities for various tissues. The
  first column shows the conditional probability that a gene has an Egl
  phenotype given that it has expression in tissue $X$ (given by the row).
  The second column shows the conditional probability that a gene has expression
  in the anatomy term X given that it has an Egl phenotype. The first 9 terms
  are the terms for which $P(\text{Egl}|X)$ is maximized. The last three terms
  are the terms which have the highest $P(X|\text{Egl})$. For clarity, the Pn.p
  cells are not shown even though $P(\text{Egl}|\text{Pn.p})\sim 0.24$.}

  \begin{tabular}{@{}lcc@{}}
  \toprule{}
  Tissue & $P(\text{Egl}| X)$ & $P(X|\text{Egl})$\\
  \bottomrule{}\\
  % Data goes here
  P7.pa & $0.30$ & $0.04$\\
  HSN & $0.27$ & $0.11$\\
  vulC & $0.27$ & $0.06$\\
  vulD & $0.26$ & $0.07$\\
  vulE & $0.25$ & $0.06$\\
  vulF & $0.24$ & $0.06$\\
  vulA & $0.24$ & $0.05$\\
  vulB2 & $0.23$ & $0.05$\\
  vulB1 & $0.22$ & $0.05$\\
  nervous system & $0.02$ & $0.72$ \\
  pharynx  & $0.00$ & $0.46$\\
  sex organ & $0.07$ & $0.41$\\
  tail & $0.05$ & $0.33$\\
  \bottomrule{}
  \end{tabular}
\label{tab:cond_probs}
\end{table*}

We calculated the conditional probability that a gene has an Egl phenotype given
that it's expression pattern includes a tissue term $X$ and we searched for the
tissue terms that maximized this probability. The list of terms that maximized this
probability reflected the results from running TEA on the subset of genes that have
an Egl phenotype. We also calculated the conditional probability that a gene has
expression in a tissue term $X$ given that it is annotated with an Egl phenotype
and we searched for terms that maximized this probability. The terms that maximized
this probability were `nervous system', `pharynx' (a body part with a lot of neurons),
`sex organ' and `tail' (a body part with neurons and hypodermis). In general, the
terms that had a high $P(\text{Egl}|X)$ did not have a high $P(X|\text{Egl})$.
Additionally, the terms that had a high $P(X|\text{Egl})$ are broad terms that
include a lot of cells, whereas the terms that had a high $P(\text{Egl}|X)$
were considerably more specific.
We conclude that the Egl phenotype arises from a small set of tissues. The Egl
phenotype can be best predicted by genes with expression patterns that include
at least one of a small number of cells (mainly vul cells, HSN). On the other
hand, answering whether the expression pattern of a gene includes a particular
anatomic region or tissue given that the gene has an Egl phenotype is hard to
do for small tissues or single cells. However, guesses about what functional system
or broad anatomic region may be affected by an Egl gene can be answered with
confidence ($\sim70\%$ of the time, the nervous system is the system
affected by an Egl mutant).

\section*{Conclusions}

We have highlighted three possible uses for phenotype and GO enrichment analyses.
Taken together, the WormBase Enrichment Suite can help guide researchers as they
search for a phenologue that is representative of a human disease that has no
immediately obvious counterpart in the worm. Such phenologues may benefit from
the fact that they represent an unbiased approach to disease modeling as long as
the human genes are selected from unbiased screens. GWAS data suggests that certain
aspects of lupus biology may be best understood from a developmental perspective
in \cel{}. On the other hand, our results suggest that rheumatoid arthritis
might not be effectively modeled in the worm, although it is possible that our
inability to find phenologues more concretely is the result of the GWAS screens:
If GWAS screens identified factors that lead to worse prognosis of rheumatoid
arthritis, but not factors that cause onset of the disease, this could explain
the lack of phenotype enrichment in \cel{}.

% Phenologues

% Screen Design

% Curation and annotation biases


The addition of GO and Phenotype Ontology enrichment testing to WormBase marks
an important step towards a unified set of analyses that can help researchers
to understand genomic datasets. By offering tissue, phenotype and gene ontology
enrichment on a single site, researchers will benefit both in the speed at which
the analysis is completed and from the fact that the analyses are all
carried out with the same methodology and the same underlying algorithms.
























\end{document}
